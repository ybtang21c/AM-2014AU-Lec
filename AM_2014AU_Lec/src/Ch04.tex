\setcounter{chapter}{3}

\chapter{导数与不定积分}

\section{导数的概念}

\subsection{函数在一点处的导数}

{\bf 定义4.1.1:}函数$y=f(x)$在$x_0$的某领域内有定义,若
$$\limdx\df{f(x_0+\dx)-f(x_0)}{\dx}$$
存在, 则称其为{\it $f(x)$在$x_0$处的导数}, 记为
$$f\,'(x_0),\quad \left.\df{\d y}{\d x}\right|_{x=x_0},
\quad y'_x|_{x=x_0}$$

{\bf 例1:}假设$f\,'(x_0)$存在,则
\begin{enumerate}[(1)]
  \setlength{\itemindent}{1cm}
  \item $\limdx\df{f(x_0-\Delta x)-f(x_0)}{\Delta
  x}=$ \underline{\quad{$-f\,'(x_0)$}\quad} 
  \item $\lim\limits_{h\to 0}\df{f(x_0+2h)-f(x_0)}{h}=$
   \underline{\quad{$2f\,'(x_0)$}\quad} 
  \item $\lim\limits_{h\to 0}\df{f(x_0+h)-f(x_0-h)}{h}=$ 
  \underline{\quad{$2f\,'(x_0)$}\quad}
\end{enumerate}

{\bf 思考:}若以上某个极限存在,是否就意味着$f(x)$在$x_0$可导?

\subsubsection{【导数的物理/几何意义】}

\begin{itemize}
  \setlength{\itemindent}{1cm}
  \item {\it 切线斜率:}
  $$k(x_0)=\lim\limits_{x\to x_0}\df{f(x)-f(x_0)}{x-x_0}$$
  \item {\it 瞬时速度:}
  $$v(t_0)=\lim\limits_{t\to t_0}\df{S(t)-S(t_0)}{t-t_0}$$
\end{itemize}

{\bf 导数:函数关于自变量的相对变化率!}\ps{当自变量发生变化时,函数值发生的相应变化与之的比率}

{\bf 例:}已知函数$f(x)$在点$x_0$处可导,求曲线$y=f(x)$在该点的切线和法线方程。

{\bf 思考:}可导等价于有切线吗?(否)

{\bf P156-例1:}讨论函数$f(x)=\sqrt[3]x$在点$x=0$是否可导?

\subsubsection{【导数存在的条件】}

{\bf 定理4.1.1:}$f(x)$在$x_0$可导,当且仅当在该点的左、右导数存在且相等。

{\bf 定理:}初等函数在其定义域内是处处可导的。

{\bf 定理4.1.2:}$f(x)$在一点可导,则一定在该点连续。

{\bf P164-例8:}确定常数$a,b$的值,使得函数
$$f(x)=\left\{\begin{array}{ll}ax+b,& x>0\\
e^x,& x\leq 0\end{array}\right.$$
在$x=0$可导。

{\bf 例:}设$f(x)=\left\{\begin{array}{ll}
\df{1-\sqrt{1-x}}{x}, & x<0,\\
a+bx, & x\geq 0
\end{array}\right.$,求$a,b$使$f(x)$处处可导。

{\bf 例:}问题讨论
\begin{enumerate} 
  \setlength{\itemindent}{1cm}
  \item 若$f(x),g(x)$在$x_0$均不可导,是否$f(x)+g(x),$ $f(x)g(x)$必不可导?
  ({$\times$}) 
  \item 若对任意$x\in (a,b)$,恒有$f(x)<g(x)$,且$f(x),g(x)$均在$(a,b)$内
  可导,问是否必有$f\,'(x)<g'(x)$? ({$\times$}) 
  \item 若$f(x)$在$\mathbb{R}$上可导,且$\limx{+\infty}f(x)=\infty$,是否
  必有$\limx{+\infty}f\,'(x)=\infty$? ({$\times$})
  \item 若$f(x)$在$(a,b)$内可导,且$\limx{a^+}f(x)=\infty$,是否
  必有$\limx{a^+}f\,'(x)=\infty$? ({$\times$}) 
  \item 若$f(x)$可导且为奇(偶)函数,则$f\,'(x)$也有奇偶性? ({$\surd$}) 
  \item 若$f(x)$可导且为周期函数,则$f\,'(x)$也是周期函数? ({$\surd$})
\end{enumerate}

\subsection{导函数}

{\bf 例}({\it 一些常用函数的导函数})
\begin{enumerate}[(1)]
  \setlength{\itemindent}{1cm}
  \item $f(x)=C\;(C\mbox{为常数})$ \hfill $f\,'(x)=0$ 
  \item $f(x)=x^n\;(n\in\mathbb{Z})$ \hfill $f\,'(x)=nx^{n-1}\,(n\ne
  0)$ 
  \item $f(x)=e^x$ \hfill $f\,'(x)=e^x$ 
  \item $f(x)=\ln x$ \hfill $f\,'(x)=\df 1x$ 
  \item $f(x)=\sin x$ \hfill $f\,'(x)=\cos x$ 
  \item $f(x)=\cos x$ \hfill $f\,'(x)=-\sin x$
\end{enumerate}

\section{导数的计算}

\subsection{四则运算的求导法则}

{\bf 定理4.2.1:}设$u(x),v(x)$均在$x$可导,则
\begin{enumerate}[(1)]
  \setlength{\itemindent}{1cm}
  \item $[u(x)\pm v(x)]'=u'(x)\pm v'(x)$ 
  \item $[u(x)v(x)]' =u'(x)v(x)+u(x)v'(x)$ 
  \item $\left[\df{u(x)}{v(x)}\right]'
  =\df{u'(x)v(x)-u(x)v'(x)}{v^2(x)}\;(v(x)\ne 0)$
\end{enumerate}

{\bf P172-例2-4:}计算以下函数的导函数
\begin{enumerate}[(1)]
  \setlength{\itemindent}{1cm}
  \item $f(x)=2x^3+3x-4x+5-\df 6x$ 
  \item $f(x)=e^x\sin x$ 
  \item $f(x)=\df{x-1}{x+1}$ 
  \item $f(x)=\df 1{\ln x}$ 
  \item $f(x)=\tan x$ \hfill $f\,'(x)=\sec^2 x$ 
  \item $f(x)=\sec x$ \hfill $f\,'(x)=\sec x\tan x$
\end{enumerate}

\subsection{反函数求导法则}

{\bf 定理4.2.2:}设$y=f(x)$是$x=\varphi(y)$的反函数,
若$x=\varphi(y)$在$y$处可导,且$\varphi'(x)\ne 0$,则
$y=f(x)$在点$x=\varphi(y)$处可导,且
$$f\,'(x)=\df{1}{\varphi'(y)}$$

{\bf P174-例7-8:}计算下列函数的导函数
\begin{enumerate}[(1)]
  \setlength{\itemindent}{1cm}
  \item $f(x)=\arcsin x$ \hfill $f\,'(x)=\df{1}{\sqrt{1-x^2}}$ 
  \item $f(x)=\arctan x$ \hfill $f\,'(x)=\df{1}{1+x^2}$
\end{enumerate}

\subsection{复合函数的求导法则}

{\bf 定理4.2.3}({\it 链式法则})设函数$u=\varphi(x)$在$x$处可导,
函数$y=f(u)$在$u=\varphi(x)$处可导,则复合函数$y=f[\varphi(x)]$
在$x$处可导,且
$$y'_x=f\,'(u)\varphi'(x)$$

{\bf P176-例5:}计算下列函数的导函数
\begin{enumerate}[(1)]
  \setlength{\itemindent}{1cm}
  \item $y=a^x\;(a>0,a\ne 1)$ \hfill $y'=a^x\ln a$ 
  \item $y=x^a$ \hfill $y'=ax^{a-1}$
\end{enumerate}

{\bf P176-例10-19:}计算下列函数的导函数
\begin{enumerate}[(1)]
  \setlength{\itemindent}{1cm}
  \item $y=e^{x^2}$ 
  \item $y=\sin (3x+2)$ 
  \item $y=\cos^2(1-2x)$ 
  \item $y=\ln\sin e^{-x}$ 
  \item $y=(1-30x)^{50}$ 
  \item $y=\ln(1+x^2)$ 
  \item $y=e^{\sqrt{1-3x}}$ 
  \item $y=x^x$ 
  \item $y=e^{\tan\frac 1x}$
\end{enumerate}

{\bf P180-例20:}设$f(x)$可导,且$f\,'\left(\df{\pi}{4}\right)=1$,求
$$\varphi(x)=f\left(\arctan\df{1+x}{1-x}\right)$$
在$x=0$处的导数。

\subsection{高阶导数}

{\bf 定义4.2.1}({\it $n$阶导数})
$$f^{\,(n)}(x)=\left[f^{\,(n-1)}(x)\right]'_x$$

{\bf P181-例34:}求函数$f(x)=x^3+2x^2-3x+10$的各阶导函数。

{\bf 注:}若$P(x)$为$n$次多项式,则$P^{(n+1)}(x)=0$。

{\bf P182-例25-26:}求以下函数的$n$阶导数
\begin{enumerate}[(1)]
  \setlength{\itemindent}{1cm}
  \item $y=\df 1x$ \hfill $y^{(n)}(x)=(-1)^n\df{n!}{x^{n+1}}$ 
  \item $y=\sin x$ \hfill
  $y^{(n)}(x)=\sin\left(\df{n\pi}{2}+x\right)$ 
  \item $y=xe^x$ \hfill $y^{(n)}(x)=(n+x)e^x$
\end{enumerate}

{\bf 【Leibnitz公式】}

$${\left[u(x)v(x)\right]^{(n)}=
\sum\limits_{k=0}^nC_n^ku^{(n-k)}(x)v^{(k)}(x)}$$

{\bf P184-例27:}设$y=x^3e^x$,求$y^{(10)}$。

{\bf 例:}设$y=\arctan x$,求$y^{(n)}(0)$

{\bf 习题集P77-例23:}设$y=(\arcsin x)^2$,求$y^{(n)}(0)$

\subsection{隐函数求导法则}

{\bf 隐函数:}由形如$f(x,y)=0$的方程所确定的函数

{\bf P185-例28:}设$y=y(x)$是由方程
$$x^3+y^3=3xy$$
所确定的隐函数,满足$y(3/2)=3/2$,求其
在点$(3/2,3/2)$处的切线方程。

\begin{center}
	\resizebox{!}{5cm}{\includegraphics{./images/ch4/x3y33xy.jpg}}
\end{center}

{\bf P186-例29:}设$y=y(x)$是由方程$y^2=x^2-\cos y$所确定的隐函数,求$y''(x)$。

\begin{center}
	\resizebox{!}{6cm}{\includegraphics{./images/ch4/y2x2-cosy.jpg}}
\end{center}

{\bf P187-例30:}求函数$y=(x^2+1)\sqrt[3]{(x-2)^2(x^2+x)}$的导数。

\subsection{参数方程求导法则}

设函数$y=y(x)$由参数方程
$$\left\{
\begin{array}{l}
x=\varphi(t)\\
y=\psi(t)
\end{array}
\right.$$
确定, $x=\varphi(t)$可逆, 则
$$y'(x)=\df{\psi'(t)}{\varphi'(t)}$$

$$y''(x)=\df{\psi''(t)\varphi'(t)-\psi'(t)\varphi''(t)}{[\varphi'(t)]^3}$$

{\bf P188-例31:}求抛物线$x=y^2$在$(1,1)$和$(4,-2)$处的切线方程。

{\bf P189-例32:}已知$\left\{\begin{array}{l}x=t-\sin t\\
y=1-\cos t\end{array}\right.$,求$y''(x)$。

\begin{center}
	\resizebox{!}{4cm}{\includegraphics{./images/ch4/sphereRoll.jpg}}
\end{center}

{\bf 例:}设$\left\{\begin{array}{l}x=f\,'(t)\\ y=tf\,'(t)-f(t)
\end{array}\right.$,求$\df{\d^2y}{\d x^2}$,其中$f\,''(x)$存在且不为零。

\section{微分}

\subsection{概念}

{\bf 【局部线性化和“以直代曲”】}

若$f(x)$在$x_0$可导,则
$$f(x)=f(x_0)+f\,'(x_0)(x-x_0)+\circ(x-x_0)\quad(x\to x_0)$$ 
即:{\bf 在$x_0$附近,$f(x)$可以近似地表示为一个线性函数} 
$$f(x)\approx f(x_0)+f\,'(x_0)(x-x_0)$$

\begin{center}
	\resizebox{!}{6cm}{\includegraphics{./images/ch4/dy.pdf}}
\end{center}

{\bf 定义4.3.1:}设$y=f(x)$在$x_0$的某领域内有定义,
若存在与$\Delta x$无关的常数$A$,使得$\Delta y=f(x_0+\Delta x)-f(x_0)$满足
$$\Delta y=A\Delta x+\circ(\Delta x)\;(\Delta x\to 0)$$ 
则称$y=f(x)$在$x_0${\it 可微}, $A\Delta x$称为{\it $y=f(x)$在$x_0$处的微分},
记为 $$\left.\d y\right|_{x=x_0}\quad \mbox{或} \quad
\left.\d f(x)\right|_{x=x_0}$$

{\bf 定理4.3.1:}设$y=f(x)$在$x_0$可微,当且仅当$y=f(x)$在$x_0$可导,且
$$\left.\d y\right|_{x=x_0}=f\,'(x_0)\d x\quad 
\mbox{或} \quad\left.\d f(x)\right|_{x=x_0}=f\,'(x_0)\d x$$

{\bf P196-例3:}设$f(x)=x^3+2x^2-3x+6$,求$\d f(x)$和$\d f(x)|_{x=1}$,
并求其在$(1,6)$处的局部线性化函数$L(x)$。

\subsection{微分的运算法则}

{\bf 定理4.3.2}(四则运算)设$u(x),v(x)$可导,则
\begin{enumerate}[(1)]
  \setlength{\itemindent}{1cm}
  \item $\d (u\pm v)=\d u\pm \d v$
  \item $\d(uv)=v\d u+u\d v$
  \item $\d\df uv=\df{v\d u-u\d v}{v^2}$
\end{enumerate}

{\bf 定理4.3.3}(复合运算)设$y=f(u),u=\varphi(x)$均可微,
则$y=f[\varphi(x)]$可微,
$$\d y=f\,'(u)\d u=f\,'(u)\varphi'(x)\d x$$

{\bf P201-例6:}求函数$y=e^{2x-1}\sin x$的微分。

{\bf P202-例7:}试将下列微分形式表示为某一函数的微分 
\begin{enumerate}[(1)]
  \setlength{\itemindent}{1cm}
  \item $x^2\d x$ 
  \item $e^{2x}\d x$ 
  \item $\cos(5x-1)\d x$ 
  \item $\df{1}{1+2x^2}\d x$
\end{enumerate}

\section{变化率和相关变化率}

{\bf 变化率:}一个变量随另一个变量变化过程中,相对于后者发生变化的速率,或者
{\it 二者的相关变化量的比值}

{\bf 导数:}对于各种变化率的数学抽象

\begin{itemize}
  \setlength{\itemindent}{1cm}
  \item {斜率}:线性函数的函数值关于自变量的变化率 
  \item {速度}:位移关于时间的变化率 
  \item {密度}:质量关于体积的变化率 
  \item {电流强度}:电量关于时间的变化率 
  \item {边际收益}:收益关于投入的变化率 
  \item {\ldots\ldots} 
\end{itemize}

{\bf P209-例7:}有一深度$8$m,上底直径$8$m的圆锥形容器,
以$4$m$^3$/min的速率向其中注水,当容器中水深$5$m时,水面上升的速度是多少?

{\bf P209-例8:}甲乙两船分别向南和向东航行。在初始时刻,甲船恰位于乙船北方
$40$km处,后来在某一时刻测得甲船向南航行了20km,此时速度为15km/h;
乙船向东航行了15km,此时速度为25km/h。问该时刻两船是在相互靠近还是远离,
二者的相对速度是多少?

\begin{shaded}
{\bf 应用题结题的一般步骤}
\begin{enumerate}
  \setlength{\itemindent}{1cm}
  \item {{\bf 画图$^*$:}}画出示意图
  \item {{\bf 确定变量:}}给出各变量的数学符号表示
  \item {{\bf 建立关系:}}根据已知,建立变量关系式
  \item {{\bf 求导:}}对所建立的关系式两边求导
  \item {{\bf 求解:}}整理新的关系式,得出结果
\end{enumerate}
\end{shaded}

\section{不定积分}

\subsection{原函数与不定积分}

{\bf 定义4.5.1:}若在区间$I$上,$f\,'(x)=f(x)$,
则称$F(x)$是{\it $f(x)$在区间$I$上的原函数}。

{\bf 性质:}若$F(x)$是$f(x)$在区间$I$上的原函数,则
\begin{enumerate}
  \setlength{\itemindent}{1cm}
  \item $F(x)+C$也是$f(x)$在区间$I$上的原函数
  \item $f(x)$的任意两个原函数只相差一个常数
\end{enumerate}

{\bf 例:}不定积分表达式的多样形式
\begin{enumerate}[(1)]
  \setlength{\itemindent}{1cm}
  \item $\dint\sin x\cos x\d x=\df12\sin^2x+C$
  \item $\dint\cos x\sin x\d x=\df12\cos^2x+C$
  \item $\dint\df12\sin2x\d x=\cos2x+C$
\end{enumerate}
以上三者都属于同一个函数族,仅相差一个常数!\ps{要想判断它们是否相同,
最可靠的方法还是求导,比较其导函数是否相同}

{\bf 例:}证明:函数$$f(x)=\left\{\begin{array}{ll}
0\;& x\ne 0\\1\;& x=0\end{array}\right.$$在$\mathbb{R}$上不存在原函数

{\bf 注:}{具有第一类间断点的函数都不存在原函数}

{\bf 定理}({\it 原函数的存在性})
若函数$f(x)$在区间$I$上连续,则在$I$上存在原函数。

{\bf 定义4.5.2:}函数$f(x)$在区间$I$上的全体原函数称为{\it $f(x)$在$I$上的不定积分},
记为
$$\int f(x)\d x$$

{\bf 注:}若$F(x)$是$f(x)$在区间$I$上的一个原函数, 则
$${\int f(x)\d x=F(x)+C},$$
其中$C$为任意常数。

{\bf P215-性质4.5.1:}
\begin{enumerate}[(1)]
  \setlength{\itemindent}{1cm}
  \item $\left(\displaystyle\int f(x)\d x\right)'=f(x)$ 
  \item $\d\left[\displaystyle\int f(x)\d x\right]=f(x)\d x$ 
  \item $\dint f\,'(x)\d x =f(x)+C$
  \item $\dint \d f(x)=f(x)+C$
\end{enumerate}

\subsection{基本不定积分公式}

{\it 求不定积分是求导的“逆运算”}
\begin{enumerate} [(1)]
  \setlength{\itemindent}{1cm}
  \item $(C)'=0$\hfill  {$\dint 0\d x=C$} 
  \item $(x^a)'=ax^{a-1}$\hfill  {$\dint x^a\d x=\df{1}{a+1}x^{a+1}+C$}
  \item $(e^x)'=e^x$\hfill  {$\dint e^x\d x=e^x+C$} 
  \item $(a^x)'=a^x\ln a$\hfill  {$\dint a^x\d x=\df{a^x}{\ln a}+C$}
  \item $(\ln x)'=\df 1x$\hfill  {$\dint \df 1x\d x=\ln|x|+C$}
  \item $(\sin x)'=\cos x$\hfill  {$\dint \cos x\d x=\sin x+C$} 
  \item $(\cos x)'=\sin x$\hfill  {$\dint \sin x\d x=-\cos x+C$}
  \item $(\tan x)'=\sec^2 x$\hfill  {$\dint \sec^2 x\d x=\tan x+C$} 
  \item $(\cot x)'=-\csc^2 x$\hfill  {$\dint \csc^2 x\d x=-\cot x+C$}
  \item $(\sec x)'=\sec x\tan x$\hfill $\dint\sec x\tan x\d x=\sec x+C$
  \item $(\csc x)'=-\csc x\cot x$\hfill $\dint\csc x\cot x\d x=-\csc x+C$
  \item $(\arcsin x)'=\df{1}{\sqrt{1-x^2}}$ \hfill  
  {$\dint{\df{1}{\sqrt{1-x^2}}}\d x=\arcsin x+C$} 
  \item $(\arctan x)'=\df{1}{1+x^2}$ \hfill 
  {$\dint \df{1}{1+x^2}\d x=\arctan x+C$}
  \item $(\cosh x)'=\sinh x$ \hfill $\dint\sinh x\d x=\cosh x+C$
  \item $(\sinh x)'=\cosh x$ \hfill $\dint\cosh x\d x=\sinh x+C$
\end{enumerate}

{\bf 注:}$x>0$时,$(\ln x)'=\df 1x$,所以$\dint\df{\d x}x=\ln x+C$,
$x<0$时,$(\ln(-x))'=\df1x$,所以$\dint\df{\d x}x=\ln(-x)+C$

{\bf P217-218:例3-4:}计算不定积分
\begin{enumerate}[(1)]
  \setlength{\itemindent}{1cm}
  \item $\dint x^2\sqrt{x}\d x$ 
  \item $\dint \df{1}{x\sqrt[3]{x}}\d x$ 
  \item $\dint \df{4^x}{9^x}\d x$ 
  \item $\dint 2^x3^{2x}5^{3x}\d x$
\end{enumerate}

\subsection{不定积分的运算法则}

{\bf 性质4.5.2}(线性运算)设函数$f(x),g(x)$的原函数存在,则
$$\int[\alpha f(x)+\beta g(x)]\d x=\alpha\int f(x)\d x+\beta\int g(x)\d x,$$
其中$\alpha,\beta$为任意常数。

{\bf 例:}计算不定积分
\begin{enumerate}[(1)]
  \setlength{\itemindent}{1cm}
  \item $\dint (4x^3-2x^2+5x+3)\d x$
  \item $\dint(1-2x)^2\sqrt x\d x$
  \item $\dint\df{(x-\sqrt x)(1+\sqrt x)}{\sqrt[3]x}\d x$
  \item $\dint\df{\d x}{\sin^2x\cos^2x}$
  \item $\dint(10^x+3\sin x+\sqrt x)\d x$
  \item $\dint\sum\limits_{k=0}^na_kx^k$
\end{enumerate}

{\bf P219-例5-6:}计算不定积分
\begin{enumerate}[(1)]
  \setlength{\itemindent}{1cm}
  \item $\dint (x^2+1)^2\d x$ 
  \item $\dint\df{(x+1)^3}{x^2}\d x$ 
  \item $\dint\df{1-x^2}{x^2(1+x^2)}\d x$
  \item $\dint \df{x^4}{1+x^2}\d x$ 
  \item $\dint\df{1}{1+\cos 2x}\d x$ 
  \item $\dint\tan^2 x\d x$
\end{enumerate}

\section{不定积分的计算}

\subsection{换元法}

{\bf 定理6.3.1}(第一换元法)设$f(u)$具有原函数,$u=\varphi(x)$可导,则
$$\dint f[\varphi(x)]\varphi'(x)\d x=\left[\dint f(u)
\d u\right]_{u=\varphi(x)}$$
{\bf 注:}设$F(x)$是$f(x)$的一个原函数,则
$$\dint f[\varphi(x)]\varphi'(x)\d x=F[\varphi(x)]+C$$

{\bf 例:}计算下列不定积分
\begin{enumerate}[(1)]
  \setlength{\itemindent}{1cm}
  \item $\dint\cos 2x\d x$ 
  \item $\dint\df 1{3+2x}\d x$
  \item $\dint\df{\d x}{x^2-a^2}$
  \item $\dint 2xe^{x^2}\d x$ 
  \item $\dint\df{x^2}{(x+2)^3}\d x$ 
  \item $\dint\df 1{a^2+x^2}\d x$ 
  \item $\dint\df 1{\sqrt{a^2-x^2}}\d x$ 
  \item $\dint \df 1{x(1+2\ln x)}\d x$ 
  \item $\dint\df {e^{3\sqrt x}}{\sqrt x}\d x$ 
  \item $\dint \sin^3x\d x$ 
  \item $\dint \sin^2x\cos^4x\d x$ 
  \item $\dint\sec^6x\d x$ 
  \item $\dint\sec x\d x$
\end{enumerate}

{\bf 注:}另解
$$\dint\sec x\d x=\dint\df{\sec x(\sec x+\tan x)}{(\sec x+\tan x)}\d x
=\dint\df{\d(\sec x+\tan x)}{(\sec x+\tan x)}=
\ln|\sec x+\tan x|+C$$

\begin{shaded}
{\bf 【第一换元法的一些常用技巧】}
\begin{enumerate}
  \setlength{\itemindent}{1cm}
  \item {\bf 分项积分:}积化和差,有理分式分解,\ldots 
  \item {\bf 降低幂次:}倍角公式,万能凑幂公式,\ldots 
    $$\dint f(x^n)x^{n-1}\d x=\df 1n\dint f(x^n)\d x^n$$ 
    $$\dint f(x^n)\df 1x\d x=\df 1n\dint f(x^n)\df{1}{x^n}\d x^n$$ 
  \item {\bf 统一函数:}三角公式,$1=\sin^2x+\cos^2x$,\ldots 
  \item {\bf 巧妙配元:}加一项减一项,\ldots
\end{enumerate}
\end{shaded}

{\bf 定理6.3.2}(第二换元法)设$x=\varphi(t)$可导且可逆,
$f[\varphi(t)]\varphi'(t)$具有原函数,则
$$\dint f(x)\d x=\left[\dint
f[\varphi(t)]\varphi'(t)\d t\right]_{t=\varphi^{-1}(x)}$$

{\bf 例:}计算下列不定积分
\begin{enumerate}[(1)]
  \setlength{\itemindent}{1cm}
  \item $\dint \sqrt{a^2-x^2}\d x$ 
  \item $\dint\df{\d x}{\sqrt{x^2+a^2}}$
  \item $\dint\df{\d x}{(x^2+a^2)^2}$
  \item $\dint\df{\d x}{\sqrt{x^2-a^2}}$ 
  \item $\dint \df{1}{1+\sqrt x}\d x$ 
  \item $\dint\df{\d x}{1+\sqrt[3]{x+2}}$ 
  \item $\dint\df{\sqrt{a^2-x^2}}{x^4}\d x$
\end{enumerate}

\begin{shaded}
{\bf 【第二换元法中的一些特殊变换】}


{\bf 【倒代换】}
当被积函数的分母次数较高,特别是含有无理项时,可尝试令
$$x=\df 1t$$

{\bf 例:}$\dint\df{\d x}{x^3\sqrt{1+x^2}}=-\dint\df{t^2\d t}{\sqrt{1+t^2}}
=-\df{\sqrt{1+x^2}}{2x^2}+\df12\ln\df{1+\sqrt{1+x^2}}{|x|}+C$

{\bf 例:}$\dint\df{\d x}{x^4\sqrt{1+x^2}}=-\dint\df{t^3\d t}{\sqrt{1+x^2}}
=-\df{\sqrt{(1+x^2)^3}}{3x^3}+\sqrt{1+x^2}x+C$

利用公式
$$\dint\df{P_n(x)}y\d x=Q_{n-1}y+\lambda\dint\df{\d x}y$$
(其中$y=\sqrt{qx^2+rx+s}$,$P_n(x),Q_{n-1}(x)$分别为$n$次和$n-1$次多项式)
可将倒代换应用于形如
$$\dint\df{\d x}{x^n\sqrt{ax^2+bx+c}}(n\geq1)$$
的不定积分。

{\bf 例:}$I=\dint\df{x^3\d x}{\sqrt{1+2x-x^2}}$

[hint:] let
$$I=(ax^2+bx+c)\sqrt{1+2x-x^2}+\lambda\dint\df{\d x}{\sqrt{1+2x-x^2}}$$
两边对$x$求导,可解得
$$a=-\df13,\quad b=-\df56,\quad,c=-\df{19}6,\quad\lambda=4$$

{\bf 例:}$I=\dint\df{\d x}{x^n\sqrt{ax^2+bx+c}},(n\geq 1)$

[hint:] let $x=\df 1t$, then
$$I=\dint\df{-t^{n-1}\d t}{\sqrt{a+bt+ct^2}}$$

形如
$$\dint\df{\sqrt{ax^2+bx+c}}{x^n}\d x$$
的积分

{\bf 【Eular代换】}

主要用于处理形如$\dint(R,\sqrt{ax^2+bx+c})$形式的积分,通过Eular代换可以将其化为
有理函数积分,具体代换形式如下:
\begin{enumerate}
  \setlength{\itemindent}{1cm}
  \item 当$a>0$时,令$\sqrt{ax^2+bx+c}\pm\sqrt ax=t$,
  即$x=\df{t^2-c}{b\pm2\sqrt at}$
  \item 当$c>0$时,令$\sqrt{ax^2+bx+c}\pm\sqrt c=tx$,
  即$x=\df{b\pm2\sqrt ct}{t^2-a}$
  \item 当$ax^2+bx+c=0$有两个实根$\alpha$和$\beta$时,
  令$\sqrt{ax^2+bx+c}=t(x-\alpha)$,
  即$x=\df{\alpha\beta-\alpha t^2}{a-t^2}$
\end{enumerate}

{\bf 例:}令$\sqrt{1+x^2}=t-x$,则
$$\dint\df{x^3}{\sqrt{1+x^2}}\d x=\dint\df{(t^2-1)^3}{8t^4}\d t$$

{\bf 例:}令$\sqrt{1+x^2}=xt-1$,则
$$\dint\df{x^3}{\sqrt{1+x^2}}\d x=-16\dint\df{t^3}{(t^2-1)^4}\d t$$
\end{shaded}

\subsection{分部积分法}

$$\dint uv'\d x=uv-\dint u'v\d x$$

{\bf 例:}计算下列不定积分
\begin{enumerate}[(1)]
  \setlength{\itemindent}{1cm}
  \item $\dint x\cos x\d x$ 
  \item $\dint xe^x\d x$ 
  \item $\dint x^2e^x\d x$ 
  \item $\dint\ln x\d x$
  \item $\dint x\ln x\d x$ 
  \item $\dint x\arctan x\d x$ 
  \item $\dint e^x\sin x\d x$
\end{enumerate}

\begin{shaded}
{\bf 【分部积分法处理原则】}
\begin{center}
	{\bf “{反对}不要碰,{三指}动一动”} 
\end{center}
将被积函数视为两个函数之积,按照{\bf{“三指幂对反”}}的次序将其中某一部分函数放到微分符号后面 
\begin{enumerate}
  \item $\sin x,\cos x$ 
  \item $e^x$ 
  \item $x^n$
\end{enumerate}
\end{shaded}

\subsection{有理函数积分}

{\bf 有理函数(有理分式):}$$f(x)=\df{P(x)}{Q(x)}$$
其中$P(x),Q(x)$均为多项式函数 ,若$P(x)$的次数小于$Q(x)$的次数,
称该函数为{\it 真分式} ,否则为{\it 假分式}

{\bf 注:}利用{\it 多项式除法},任意假分式都可以表示成一个多项式与一个真分式的和,例如:
$$\df{2x^4+x^2+3}{x^2+1}=2x^2-1+\df{4}{x^2+1}$$

\begin{shaded}
{\bf 【有理函数分解的一般过程】}

第一步:任意多项式$Q(x)$在实数系内总能分解为一个常数,与形如$(x-a)^n$与$(x^2+px+q)^m$
的诸因式之乘积,其中$a$是$Q(x)$的$n$重根,二次多项式$x^2+px+q$没有实根($p^2-4q<0$),
有共轭复根,且重数均为$m$,故
$$Q(x)=\prod\limits_{i=1}^s(x-a_i)^{\lambda_i}
\prod\limits_{j=1}^t(x^2+p_jx+q_j)^{\mu_j},$$
其中$\lambda_1,\ldots,\lambda_s,\mu_1,\ldots,\mu_t$均为正整数。

第二步:根据代数的分项分式定理,真分式$\df{P(x)}{Q(x)}$可进行如下分解
\begin{eqnarray*}
	\df{P(x)}{Q(x)}&=&\df{A_1}{(x-a_1)^{\lambda_1}}
	+\df{A_2}{(x-a_1)^{\lambda_1-1}}+\ldots
	+\df{A_{\lambda_1}}{x-a_1}+\ldots\\
	&&+\df{B_1}{(x-a_s)^{\lambda_s}}
	+\df{B_2}{(x-a_s)^{\lambda_s-1}}+\ldots
	+\df{B_{\lambda_s}}{x-a_s}+\ldots\\
	&&+\df{M_1x+N_1}{(x^2+p_1x+q_1)^{\mu_1}}
	+\df{M_2x+N_2}{(x^2+p_1x+q_1)^{\mu_1-1}}+\ldots
	+\df{M_{\mu_1}x+N_{\mu_1}}{x^2+p_1x+q_1}+\ldots\\
	&&+\df{U_1x+V_1}{(x^2+p_tx+q_t)^{\mu_t}}
	+\df{U_2x+V_2}{(x^2+p_tx+q_t)^{\mu_t-1}}+\ldots
	+\df{U_{\mu_t}x+V_{\mu_t}}{x^2+p_tx+q_t}
\end{eqnarray*}

第三步:用待定系数法求出以上的所有常数。
\end{shaded}

{\bf 例}(基本的有理函数积分)
\begin{enumerate}[(1)]
  \setlength{\itemindent}{1cm}
  \item $\dint\df A{x-a}\d x=A\ln|x-a|+C$ 
  \item $\dint\df{B\d x}{(x-a)^n}=\df B{(1-n)(x-a)^{n-1}}+C$ 
  \item $\dint\df{Cx+D}{x^2+px+q}\d x=
  \df C2\ln(x^2+px+q)+\df{2D-Cp}{\sqrt{4q-p^2}}
  \arctan\df{2x+p}{\sqrt{4q-p^2}}+C$
  \item $\dint\df{Ex+F}{(x^2+px+q)^m}\d x
  =\df E{2(1-m)(x^2+px+q)^{m-1}}+J_m$,其中:记$u^2=q-p^2/4$,
  $$J_m=\df{y}{2u^2(m-1)(y^2+u^2)^{m-1}}+
  \df{3-2m}{2u^2(1-m)}J_{m-1},$$
  $$J_1=\df1u\arctan\df yu+C$$
\end{enumerate}

{\bf 性质:}
\begin{enumerate}
  \setlength{\itemindent}{1cm}
  \item 设$Q(x)$可分解为两个没有共因式的多项式$Q_1(x),Q_2(x)$的乘积,
  则真分式$\df{P(x)}{Q(x)}$必可分解为
  两个真分式$\df{P_1(x)}{Q_1(x)},\df{P_2(x)}{Q_2(x)}$的和
  \item 任意多项式都可分解为形如$(x^2+px+q)^l$,$(x-a)^k$的多项式的乘积
\end{enumerate}

{\bf 例:}计算下列不定积分
\begin{enumerate}[(1)]
  \setlength{\itemindent}{1cm}
  \item $\dint\df{x+1}{x^2+5x+6}\d x$ 
  \item $\dint\df{x+2}{(2x+1)(x^2+x+1)}\d x$
  \item $\dint\df{x-3}{(x-1)(x^2-1)}\d x$
\end{enumerate}

{\bf 【一些可以化成有理函数的积分】}

{\bf 情形一:}
$$\dint R(\sin x,\cos x)\d x$$
令$t=\tan\df x2$(万能代换),则
$$\sin x=\df{2t}{1+t^2},\quad\cos x=\df{1-t^2}{1+t^2}$$

{\bf 例:}$\dint\df{1+\sin x}{\sin x(1+\cos x)}\d x$

{\bf 情形二:}
$$\dint R(\sin^2x,\cos^2x,\sin x,\cos x)\d x$$
令$t=tan x$

{\bf 例:}$\dint\df{\d x}{a^2\sin^2x+b^2\cos^2x},\;(ab\ne 0)$

{\bf 情形三:}
$$\dint R\left(x,\sqrt[n]{\df{ax+b}{cx+d}}\right),(ad-bc\ne0),$$
令$t=\sqrt[n]{\df{ax+b}{cx+d}}$

{\bf 例:}$\dint\df1x\sqrt{\df{x+2}{x-2}}\d x$

{\bf 例:}$\dint\df{\d x}{(1+x)\sqrt{2+x-x^2}}$

[hint:] let $t=\sqrt{\df{1+x}{2-x}}$, hence $x=\df{2t^2-1}{1+t^2}$
$$\dint\df{\d x}{(1+x)\sqrt{2+x-x^2}}=\dint\df2{3t^2}\d t$$

{\bf 情形四:}
$$\dint R(x,\sqrt{ax^2+bx+c})\d x,\;(a>0,b^2-4ac\ne0)$$
令$u=x+\df b{2a},k^2=\left|\df{4ac-b^2}{4a^2}\right|$,则$ax^2+bx+c$
必为如下三种情形之一
$$|a(u^2+k^2)|,\quad a|u^2-k^2|,\quad a|k^2-u^2|,$$
从而上述无理根式的积分可化为下列三种形式之一
$$\dint R(u,\sqrt{u^2\pm k^2})\d u,\quad\dint R(u,\sqrt{k^2-u^2})\d u,$$
分别令$u=k\tan t,u=k\sec t,u=k\sin t$即可。

{\bf 例:}$\dint\df{\d x}{x\sqrt{x^2-2x-3}}=\df2{\sqrt3}
\arctan\df{\sqrt{x^2-2x-3}}{\sqrt3(x+1)}+C$

[hint:]let $\sqrt{x^2-2x-3}=x-t$, then $I=\df2{\sqrt3}\arctan
\df{\sqrt{x^2-2x-3}-x}{\sqrt3}+C$

{\bf 注:}$\arctan\df{\sqrt{x^2-2x-3}}{\sqrt3(x+1)}-\df{\pi}3=\arctan
\df{\sqrt{x^2-2x-3}-x}{\sqrt3}$,若令$\sqrt{x^2-2x-3}=x+t$,会有类似的效果

{\bf 例}(一些可以化为有理函数的积分)
\begin{enumerate}[(1)]
  \setlength{\itemindent}{1cm}
  \item $\dint\df{1+\sin x}{\sin x(1+\cos x)}\d x$ 
  \item $\dint\df{\sqrt{x-1}}{x}\d x$ 
  \item $\dint\df{\d x}{1+\sqrt[3]{x+2}}$ 
  \item $\dint\df{\d x}{(1+\sqrt[3]{x})\sqrt x}$ 
  \item $\dint\df 1x\sqrt{\df{1+x}{x}}\d x$
\end{enumerate}

{\bf 例:}计算下列不定积分
\begin{enumerate}
  \item $\dint\df{x^2+\sin^2x}{x^2+1}\sec^2x\d x$
  \item $\dint\df 1{(x^2+1)\sqrt{1-x^2}}\d x$
  \item $\dint\df{3\cos x-4\sin x}{\cos x+2\sin x}\d x$
\end{enumerate}

\newpage

\section*{课后作业}

\begin{itemize}
  \item 习题4.1:6,7,8,9,10
  \item 证明过曲线$\sqrt x+\sqrt y=\sqrt a$上任一点$(x_0,y_0)$的切线在两坐标轴上
		的截距之和为常数。
  \item 习题4.2:1,4,6,11,12,10,11,14,15,18,20
  \item 求函数$y=\df{1}{x^2-3x-4}$的$n$阶导函数。
  \item 习题4.3:3,4,8,11
  \item 习题4.4:6,9,12,14,15,17
  \item 习题4.5:1,4,5,6
  \item 习题6.3:4-6
\end{itemize}

{\bf 【课堂练习与思考题】}

\begin{itemize}
  \item 习题4.1:12,13,14,15,16
  \item 习题4.2:5,8,9,16,17,19,22-24
  \item 习题4.3:2,9,10,12-14
  \item 习题4.4:16,17,19
  \item 习题4.5:7-12
\end{itemize}