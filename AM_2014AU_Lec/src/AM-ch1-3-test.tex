\begin{center}
	{\Large\bf 数列极限、级数与函数极限测验}
	
	(时间:120分钟)
\end{center}

{\bf 一、计算以下极限(每题4分):}
\begin{enumerate}[(1)]
  \setlength{\itemindent}{1cm}
  \item $\limn\sqrt[n]{a^n+b^n},\;(a,b>0)$
  \item $\limn\df{1+2!+\ldots+n!}{n!}$
  \item $\limn\left[\df1{1\cdot 2\cdot 3}+\df1{2\cdot 3\cdot 4}+\ldots+
  \df1{n\cdot (n+1)\cdot (n+2)}\right]$
  \item $\limn\df{n^5}{2^n}$
  \item $\limn\df{1+2^3+\ldots+n^3}{n^4}$
%   \item $\limx{0}\df{\tan x-\sin x-x^3}{\sin^3x}$
  \item $\limx{\infty}\left[\df{x^2}{(x-a)(x-b)}\right]^x$
  \item $\limx{+\infty}(\sqrt{x^2+x}-\sqrt[3]{x^3+x^2})$ 
  \item $\limx{0}\df{1-\cos x\cos 2x}{x^2}$
%   \item $\limx{+\infty}\df{\ln(2+3e^{3x})}{\ln(3+2e^{2x})}$
  \item $\limx{0}\df1{x^3}\left[\left(\df{2+\cos x}3\right)^x-1\right]$
%   \item $\limx{\pi/4}(\tan x)^{\tan 2x}$
  \item $\limx{0}\df{\arctan x-\sin x}{x^3}$
\end{enumerate}

{\bf 二、判定以下级数的敛散性(每题4分):}
\begin{enumerate}[(1)]
  \setlength{\itemindent}{1cm}
  \item $\sum\limits_{n=2}^{\infty}\df1{\ln n^{\ln n}}$
  \item $\sumn\sin(\pi\sqrt{n^2-1})$
  \item $\sumn(-1)^n\df{n^{n+\frac1n}}{\left(n+\df1n\right)^n}$
  \item $\sumn\df{n!}{{(\sqrt5+1)}{(\sqrt5+2)}\ldots{(\sqrt5+n)}}$
  \item $\sumn\df{2^nn!}{n^n}$
\end{enumerate}

{\bf 三、用定义证明以下极限(每题5分):}
\begin{enumerate}[(1)]
  \setlength{\itemindent}{1cm}
  \item $(n+1)^k-n^k\to 0,\;(n\to\infty)$,其中$0<k<1$
  \item $\limx{1}\df1{x}=1$
\end{enumerate}

{\bf 四、证明(6分):}设$a_n>0\;(n=1,2,\ldots)$,记$r_n=\df{a_{n+1}}{a_n}$,
\begin{enumerate}[(1)]
  \setlength{\itemindent}{1cm}
  \item 若对充分大的$n$,总有$r_n\geq 1$,则$\sumn a_n$发散;
  \item 若存在常数$r\in(0,1)$,使对充分大的$n$,总有$r_n\leq r$,则$\sumn a_n$收敛。
\end{enumerate}

{\bf 五、证明(12分):}
\begin{enumerate}[(1)]
  \setlength{\itemindent}{1cm}
  \item 方程$x^n+x^{n-1}+\ldots+x=1\;(n=2,3,\ldots)$在$\left(\df12,1\right)$
  内有且仅有一个实根;
  \item 记(1)中的实根为$x_n$,证明$\limn x_n$存在,并求此极限。
\end{enumerate}

% {\bf 五、证明(6分):}已知函数$f(x)$在$\mathbb{R}$上连续,且$\limx{\infty}f(x)$存在,
% 则$f(x)$在$\mathbb{R}$上有界。
% 
% {\bf 六、证明(6分):}$f(x)=xD(x)$只在$x=0$处连续,其中$D(x)$为Dirichlet函数。

{\bf 六、证明(12分):}
\begin{enumerate}[(1)]
  \setlength{\itemindent}{1cm}
  \item 若$a_n>0\,(n\in\mathbb{N})$,$\limn a_n=a$,
  则$\limn\sqrt{a_n}=\sqrt a$;
  \item 若$a_1>0$,$a_{n+1}=a_n+\df 1{a_n},\,(n=1,2,\ldots)$,则
  $$\limn\df{a_n}{\sqrt{2n}}=1.$$
\end{enumerate}

\newpage

\begin{center}
	{\Large\bf 解答与评分标准}\ps{1.若解法与参考答案不同,参照本评分标准的特点分段给分\\
	2.计算题只写结果,缺少计算过程,最多得1分}
\end{center}

{\bf 一、解:}
\begin{enumerate}[(1)]
  \setlength{\itemindent}{1cm}
  \item 不妨设$a\leq b$,则
  $$b<\sqrt[n]{a^n+b^n}\leq\sqrt[n]2b,\eqno{{(+2\;\mbox{分})}}$$
  由于$\sqrt[n]2\to 1(n\to\infty)$,故由夹逼定理,原式$=b=\max\{a,b\}$.\hfill{{(+2分)}}
  \item 由于
  $$\limn\df{(n+1)!}{(n+1)!-n!}=\limn\df{(n+1)!}{n\cdot
  n!}=\limn\df{n+1}{n}=1,\eqno{{(+2\;\mbox{分})}}$$ 
  故由Stolz定理,原式$=1$.\hfill{{(+2分)}}
  \item 
  \begin{align}
  	\mbox{原式}&=\df12\limn\sum\limits_{k=1}^n\left(\df 1n-\df 2{n+1}+\df
  	1{n+2}\right)\tag{{+2\;\mbox{分}}}\\ 
  	&=\df12\limn\left(1-2\cdot\df 1{2}+\df 13+\df 12-2\cdot
  	\df13+\df14+\ldots+\df 1n-2\cdot\df 1{n+1}+\df 1{n+2}\right)\notag\\
  	&=\df12\limn\left(1-\df12-\df1{n+1}+\df 1{n+2}\right)=\df14\tag{{+2\;\mbox{分}}}
  \end{align}
  \item
  记$a_n=\df{n^5}{2^n}$,显然$a_n>0\,(n\in\mathbb{N})$。又当$n>\df{1}{\sqrt[5]2-1}$时,
  $$\df{a_{n+1}}{a_n}=\left(\df{n+1}{n}\right)^5\cdot\df 12<1,\eqno{(+2\;\mbox{分})}$$
  故由单调有界原理,$\{a_n\}$收敛,不妨设其极限为$a$。对递推式
  $$a_{n+1}=\left(\df{n+1}{n}\right)^5\cdot\df 12a_n$$
  两端同时取极限,可得$a=\df 12a$,从而可知$\limn a_n=a=0$。\hfill{{(+2分)}}
  \item 由Stolz定理,
  \begin{align}
  	\mbox{原式}&=\limn\df{n^3}{n^4-(n-1)^4}\tag{{+2\;\mbox{分}}}\\
  	&=\limn\df{n^3}{C_4^1n^3-C_4^2n^2+C_4^3n-1}=\df14\tag{{+2\;\mbox{分}}}
  \end{align}
%   \item 
%   \begin{align}
%   	\mbox{原式}&=\limx{0}\df{\tan x(1-\cos x)-x^3}{x^3}
%   	=\limx{0}\df{\tan x(1-cos x)}{x^3}-1\tag{{+2\;\mbox{分}}}\\
%   	&=\limx{0}\df{\df12x^3}{x^3}-1=-\df12\tag{{+2\;\mbox{分}}}
%   \end{align}
  \item
  \begin{align}
  	\mbox{原式}&=\limx{\infty}\left\{\left[1+\df{(a-b)x+ab}
  	{(x-a)(x+b)}\right]^{\frac{(x-a)(x+b)}{(a-b)x+ab}}\right\}
  	^\frac{[(a-b)x+ab]x}{(x-a)(x+b)\tag{{+2\;\mbox{分}}}}\\
  	&=\exp\left\{\df{[(a-b)x+ab]x}{(x-a)(x+b)}\ln\left[
  	1+\df{(a-b)x+ab}{(x-a)(x+b)}\right]^{\frac{(x-a)(x+b)}
  	{(a-b)x+ab}}\right\}=e^{a-b}\tag{{+2\;\mbox{分}}}
  \end{align}
  \item 
  \begin{align}
  	\mbox{原式}&=\limx{+\infty}(x^3+x^2)^{\frac13}
  	\left[\left(\df{x+1}x\right)^{\frac16}-1\right]\tag{{+1\;\mbox{分}}}\\
  	&=\limx{+\infty}\left(1+\df1x\right)^{\frac13}\df16
  	\ln\left(\df{x+1}x\right)^x\tag{{+2\;\mbox{分}}}\\
  	&=\df16\tag{{+1\;\mbox{分}}}
  \end{align}
  \item
  \begin{align}
  	\mbox{原式}&=\limx{0}\df{(1-\cos x)+\cos
  	x(1-\cos2x)}{x^2}\tag{{+2\;\mbox{分}}}\\
  	&=\limx{0}\df{1-\cos x}{x^2}+
  	\limx{0}\cos x\df{1-\cos2x}{x^2}
  	=\df12+\limx{0}\df{2x^2}{x^2}=\df52\tag{{+2\;\mbox{分}}}
  \end{align}
%   \item
%   \begin{align}
%   	\mbox{原式}&=\limx{+\infty}\df{3+\ln(2e^{-3x}+1)}
%   	{2+\ln(3e^{-2x}+1)}=\df32\tag{{+4\;\mbox{分}}}
%   \end{align}
  \item
  \begin{align}
  	\mbox{原式}&=\limx{0}\df1{x^3}\left[\exp\left(
  	x\ln\df{2+\cos x}{3}\right)-1\right]\notag\\
  	&=\limx{0}\df{x\ln\df{2+\cos x}3}{x^3}\tag{{+1\;\mbox{分}}}\\
  	&=\limx{0}\df{\cos x-1}{3x^2}\tag{{+1\;\mbox{分}}}\\
  	&=-\df16\tag{{+2\;\mbox{分}}}
  \end{align}
%   \item
%   \begin{align}
%   	\mbox{原式}&=\limx{\pi/4}[1-(1-\tan x)]^{\frac1{1-\tan
%   	x}(1-\tan x){\tan2x}}\tag{{+2\;\mbox{分}}}\\
%   	&=\limx{\pi/4}\exp\left\{\df{2\tan x}{1+\tan x}
%   	\ln[1-(1-\tan x)]^{\frac1{1-\tan x}}\right\}=e\tag{{+2\;\mbox{分}}}
%   \end{align}
  \item 令$y=\arctan x$
  \begin{align}
  	\mbox{原式}&=\lim\limits_{y\to 0}\df{y-\df
  	y{\sqrt{1+y^2}}}{y^3}\tag{{+2\;\mbox{分}}}\\
  	&=\lim\limits_{y\to 0}\df{(\sqrt{1+y^2}-1)}{y^2\sqrt{1+y^2}}
  	=\lim\limits_{y\to 0}\df1{\sqrt{1+y^2}+1}=\df12\tag{{+2\;\mbox{分}}}
  \end{align}
\end{enumerate}

{\bf 二、解:}
\begin{enumerate}[(1)]
  \setlength{\itemindent}{1cm}
  \item 当$n>e^{e^2}$时,级数各项均非负,且
  $$\df1{\ln n^{\ln n}}=\df1{n^{\ln\ln n}}<\df1{n^2}.$$
  由比较判别法,因为$\sumn\df1{n^2}$收敛,故原级数收敛。\hfill{{(+4分)}}
  \item 
  $$\sin(\pi\sqrt{n^2-1})=(-1)^{n+1}\sin(n-\sqrt{n^2-1})\pi
  =(-1)^{n+1}\sin\df{\pi}{n+\sqrt{n^2-1}},$$
  注意到$\left\{\sin\df{\pi}{n+\sqrt{n^2-1}}\right\}$单调递减趋于零,
  故由交错级数的Leibniz判别法,原级数收敛。又
  \hfill{{(+2分)}}
  $$n\sin\df{\pi}{n+\sqrt{n^2-1}}\to\df{\pi}2,\;(n\to\infty),$$
  故由p-判别法,$\sumn|\sin(\pi\sqrt{n^2-1})|$发散。
  综上,原级数是条件收敛的。\hfill{{(+2分)}}
  \item
  注意到
  $$\limn\left(1+\df1{n^2}\right)^n=\limn\exp\left[\df1n
  \ln\left(1+\df1{n^2}\right)^{n^2}\right]=1\eqno{(+2\;\mbox{分})}$$
  故
  $$\df{n^{n+\frac1n}}{\left(n+\df1n\right)^n}=\df{n^{\frac1n}}
  {\left(1+\df1{n^2}\right)^n}\to 1,\;(n\to\infty).$$
  原级数不满足级数收敛的必要条件,故发散。\hfill{{(+2分)}}
  \item 注意到
  $$\df{n!}{{(\sqrt5+1)}{(\sqrt5+2)}\ldots{(\sqrt5+n)}}
  <\df{n!}{3\cdot4\cdot5\cdots(2+n)}=\df2{(n+1)(n+2)}<\df2{n^2},$$
  级数$\sumn\df1{n^2}$收敛,由比较判别法,原级数收敛。\hfill{{(+4分)}}
  \item 记$a_n=\df{2^nn!}{n^n},(n=1,2,\ldots)$,
  $$\df{a_{n+1}}{a_n}=\df2{\left(1+\df1n\right)^n}\to\df2e,
  \;(n\to\infty).$$
  因为$2/e<1$,故由比值判别法,原级数收敛。\hfill{{(+4分)}}
\end{enumerate}

{\bf 三、证:}\ps{未用定义证明不得分!}
\begin{enumerate}[(1)]
  \setlength{\itemindent}{1cm}
  \item 对$\forall\e>0$,令$N=\e^{\frac1{k-1}}$,则对$\forall n>N$,
  有\hfill{{(+3分)}}
  $$|(n+1)^k-n^k|=n^k\left[\left(1+\df1n\right)^k-1\right]
  <n^k\left(1+\df1n-1\right)=\df1{n^{1-k}}<\df1{N^{1-k}}=\e,$$
  即证。\hfill{{(+2分)}}
  \item 对$\forall\e>0$,令$\delta=\min\{1/2,\e\}$,则对$0<|x-1|<\delta$,
  有\hfill{{(+3分)}}
  $$\left|\df1{x}-1\right|=\df{|x-1|}{|x|}<\delta\leq\e,$$
  即证。\hfill{{(+2分)}}
\end{enumerate}

{\bf 四、证:}
\begin{enumerate}[(1)]
  \setlength{\itemindent}{1cm}
  \item 由已知,存在$N\in\mathbb{N}$,对$\forall n>N$,都有$r_n\geq1$,从而
  $$a_n\geq a_{n-1}\geq\ldots\geq a_N.\eqno{(+2\;\mbox{分})}$$
  由数列极限的保号性,易知$\{a_n\}$不可能以$0$为极限,不满足级数收敛的必要条件,故原级数发散。
  \hfill{{(+1分)}}
  \item 由已知,存在$N\in\mathbb{N}$,对$\forall n>N$,都有$r_n\leq r$,从而
  $$a_n<ra_{n-1}<\ldots<r^{n-N}a_N.\eqno{(+2\;\mbox{分})}$$
  当$r<1$时,几何级数$\sum\limits_{n=N}^{\infty}r^{n-N}a_N$收敛,故
  由比较判别法,原级数收敛。\hfill{{(+1分)}}
\end{enumerate}

{\bf 五、证:}
\begin{enumerate}[(1)]
  \setlength{\itemindent}{1cm}
  \item 设
  $$P_n(x)=x^n+x^{n-1}+\ldots+x-1,\;(n=2,3,\ldots).$$
  注意到$P_n(x)$在$\left[\df12,1\right]$上连续,且$P_n(1)=n-1>0$,
  $$P_n\left(\df12\right)=\df1{2^n}+\df1{2^{n-1}}+\ldots+\df12-1<0,$$
  故由介值定理,必存在$x_n\in\left(\df12,1\right)$,使得$P_n(x_n)=0$。\hfill{{(+3分)}}
  
  又注意到$P_n(x)$在$\left(\df12,1\right)$内是严格单调递增的,故以上的$x_n$唯一。
  \hfill{{(+2分)}}
  \item 由(1)可知,$0<\df12<x_n<1,\;n=2,3,\ldots$。又由$P_n(x_n)=
  P_{n+1}(x_{n+1})=0$,
  $$P_n(x_n)-P_n(x_{n+1})=(x_n^n+x_n^{n-1}+\ldots+x_n-1)
  -(x_{n+1}^n+x_{n+1}^{n-1}+\ldots+x_{n+1}-1)=x_{n+1}^{n+1}>0,$$
  由$P_n(x)$的单调性可知$\{x_n\}$是单调递减的。由单调有界原理,
  ${x_n}$收敛。\hfill{{(+4分)}}
  
  设$a=\limn x_n$。注意到$n>2$时,$0<x_n<x_2<1$,所以$0<x_n^n<x_2^n$,由夹逼定理可知,
  $\limn x_n^n=0$。又等式\ps{由$0<x_n<1$直接推出$\limn x_n^n=0$,本步不得分!}
  $x_n^n+x_n^{n-1}+\ldots+x_n=1$
  即为
  $$\df{x_n(1-x_n^n)}{1-x_n}=1,$$
  两边同时取极限,可得$\df a{1-a}=1$,进而解得$a=\df12$,即为所求。\hfill{{(+3分)}}
 \end{enumerate}
 
% {\bf 五、证:}由$\limx{\infty}f(x)$存在,可知存在$M_1>0$和$X>0$,对任意的$|x|>X$,
% $$|f(x)|\leq M_1.\eqno{(+2\;\mbox{分})}$$
% 又$f(x)$在$\mathbb{R}$上连续,故也在$[-X,X]$上连续,从而在$[-X,X]$上有界,也即存在
% $M_2>0$,对任意$x\in[-X,X]$,
% $$|f(x)|\leq M_2.\eqno{(+2\;\mbox{分})}$$
% 令$M=\max\{M_1,M_2\}$,则对任意$x\in\mathbb{R}$,都有$|f(x)|\leq M$,
% 即证。\hfill{{(+2分)}}
% 
% \bigskip
% {\bf 六、证:}先证$xD(x)$在$x=0$处连续。注意到$f(0)=0$,故只需证明$\limx{0}xD(x)=0$。
% 事实上,对$\forall\e>0$,取$\delta=\e$,则对任意$0<|x|<\delta$,都有
% $$|xD(x)|\leq|x|<\delta=\e.$$
% 即证。\hfill{{(+3分)}}
% 
% 下证对任意$x_0\ne 0$,$xD(x)$均不连续。事实上,若取$\{x_n^{(1)}\},\{x_n^{(2)}\}$
% 分别为趋于$x_0$的一个有理数列和一个无理数列,则
% $$\limn f(x_n^{(1)})=\limn x_n^{(1)}=x_0,$$
% $$\limn f(x_n^{(2)})=\limn 0=0$$
% 故由Heine定理,$xD(x)$在$x_0$处极限不存在,从而不连续。\hfill{{(+3分)}}

% \bigskip
{\bf 六、证:}
\begin{enumerate}[(1)]
  \setlength{\itemindent}{1cm}
  \item 由极限的保号性可知,$a\geq 0$。\hfill{{(+1分)}}

若$a=0$。对$\forall\e>0$,令$\e_1=\e^2$。由$\limn a_n=0$,对$\e_1$,$\exists
N>0$,对$\forall n>N$,有 $$|a_n-0|=a_n<\e_1,$$
从而
$$|\sqrt{a_n}-0|=\sqrt{a_n}<\sqrt{\e_1}=\e,$$
也即$\limn\sqrt{a_n}=0=\sqrt a$。\hfill{{(+2分)}}

若$a>0$。由$\limn a_n=a$,对$\forall\e>0$,$\exists N>0$,对$\forall n>N$,有
$$|a_n-a|<\e,$$
进而
$$|\sqrt{a_n}-\sqrt a|=\df{|a_n-a|}{\sqrt{a_n}+\sqrt a}
<\df {|a_n-a|}{\sqrt a}<\df 1{\sqrt a}\e,$$
从而$\limn\sqrt{a_n}=\sqrt a$。\hfill{{(+3分)}}
  \item 显然$\{a_n\}$为严格单调递增的正数列。假设其收敛于$a$,在递推公式两边取极限可得
$$a=a+\df1a,$$
该方程无解,故假设不成立。由此可知$\{a_n\}$必严格单调递增趋于$+\infty$,也即
$$\limn\df{1}{a_n}=0.\eqno{(+3\;\mbox{分})}$$
从而,由Stolz定理
$$\limn\df{a_n^2}{2n}=\limn\df{a_{n+1}^2-a_n^2}{2}
=\df12\limn\left(2+\df1{a_n^2}\right)=1.$$
再由(1)的结论,即证。\hfill{{(+3分)}}
\end{enumerate}